% quantum-commands.tex
% --- Created: December 2, 2025 ---

% =================================================================
% I. BASIC MATH FIELDS AND SPACES
% =================================================================

% A. Basic Fields (R, C, N, Z, Q)
\newcommand{\R}{\mathbb{R}} % Real numbers
\newcommand{\C}{\mathbb{C}} % Complex numbers
\newcommand{\N}{\mathbb{N}} % Natural numbers
\newcommand{\Z}{\mathbb{Z}} % Integers
\newcommand{\Q}{\mathbb{Q}} % Rational numbers

% B. Vector Spaces (R^n, C^n, etc.)
\newcommand{\Rn}{\mathbb{R}^n} % n-dim Real space
\newcommand{\Cn}{\mathbb{C}^n} % n-dim Complex space
\newcommand{\Nn}{\mathbb{N}^n}
\newcommand{\Zn}{\mathbb{Z}^n}
\newcommand{\Qn}{\mathbb{Q}^n}

\newcommand{\Rm}{\mathbb{R}^m} % n-dim Real space
\newcommand{\Cm}{\mathbb{C}^m} % n-dim Complex space

% C. Matrix Spaces (R^{m x n}, C^{n x n}, etc.)
\newcommand{\Rmn}{\mathbb{R}^{m \times n}} % Real m x n matrices
\newcommand{\Cmn}{\mathbb{C}^{m \times n}} % Complex m x n matrices
\newcommand{\Rnn}{\mathbb{R}^{n \times n}} % Real n x n matrices
\newcommand{\Cnn}{\mathbb{C}^{n \times n}} % Complex n x n matrices

\newcommand{\spanop}{\operatorname{span}}
\newcommand{\dimop}{\operatorname{dim}}
\newcommand{\trop}{\operatorname{tr}}

\newcommand{\mn}{{m \times n}}
\newcommand{\nn}{{n \times n}}
\newcommand{\nm}{{n \times m}}


% =================================================================
% II. VECTORS AND SCALAR OPERATIONS
% =================================================================

% A. Vector Symbols (Bolded)
\newcommand{\zerovec}{\mathbf{0}} % Zero vector
\newcommand{\vvec}{\mathbf{v}}    % Vector v
\newcommand{\uvec}{\mathbf{u}}    % Vector u
\newcommand{\wvec}{\mathbf{w}}    % Vector w
\newcommand{\rvec}{\mathbf{r}}    % Vector r
\newcommand{\cvec}{\mathbf{c}}    % Vector c (sometimes used for coefficient vector)

% B. Norms and Unit Vectors (Requires amsmath and maybe physics package for norm command)
\newcommand{\vnorm}{\norm{\vvec}}  % Norm of v
\newcommand{\unorm}{\norm{\uvec}}  % Norm of u
\newcommand{\unitv}{\hat{\vvec}}  % Unit vector v
\newcommand{\unitu}{\hat{\uvec}}  % Unit vector u

% C. Standard Basis Vectors (i, j, k)
\newcommand{\ibase}{\mathbf{\hat{i}}} % i-hat basis vector
\newcommand{\jbase}{\mathbf{\hat{j}}} % j-hat basis vector
\newcommand{\kbase}{\mathbf{\hat{k}}} % k-hat basis vector

% =================================================================
% III. COMMON MATH OPERATIONS
% =================================================================

% A. Calculus/Derivatives
\newcommand{\od}[2]{\frac{d#1}{d#2}} % Ordinary derivative
\newcommand{\pd}[2]{\frac{\partial#1}{\partial#2}} % Partial derivative

% B. Summation
\newcommand{\sumonen}{\sum_{i=1}^{n}} % Sum from 1 to n
\newcommand{\sumzeron}{\sum_{i=0}^{n}} % Sum from 0 to n
\newcommand{\sumoneinf}{\sum_{i=1}^{\infty}} % Sum from 1 to infinity
\newcommand{\sumzeroinf}{\sum_{i=0}^{\infty}} % Sum from 0 to infinity

% C. Vector Algebra
\newcommand{\innerprod}[2]{\left\langle#1 \cdot #2 \right\rangle} % Standard inner product (Dot product notation)
\newcommand{\projection}[2]{\text{proj}_{#1}{#2}} % Vector projection

% D. Conjugation/Adjoint
\newcommand{\conj}[1]{{#1}^*} % Complex conjugate (Often used for scalar or matrix elements)
\newcommand{\inverse}[1]{{#1}^{-1}}

% =================================================================
% IV. QUANTUM MECHANICS / BRA-KET NOTATION
% =================================================================

% A. Arbitrary States (Phi, Psi, Chi)
\newcommand{\ketphi}{\ket{\phi}} % Ket phi
\newcommand{\braphi}{\bra{\phi}} % Bra phi
\newcommand{\ketpsi}{\ket{\psi}} % Ket psi
\newcommand{\brapsi}{\bra{\psi}} % Bra psi
\newcommand{\ketchi}{\ket{\chi}} % Ket chi
\newcommand{\brachi}{\bra{\chi}} % Bra chi

\newcommand{\ketu}{\ket{u}}
\newcommand{\ketv}{\ket{v}}
\newcommand{\brau}{\bra{u}}
\newcommand{\brav}{\bra{v}}

% B. Common Single-Qubit States
\newcommand{\ketzero}{\ket{0}}   % Ket 0 (Computational basis)
\newcommand{\ketone}{\ket{1}}    % Ket 1 (Computational basis)
\newcommand{\ketplus}{\ket{+}}   % Ket + (Hadamard basis)
\newcommand{\ketminus}{\ket{-}}  % Ket - (Hadamard basis)

% C. Bell States (Maximally Entangled Two-Qubit States)
\newcommand{\bellphi}{\ket{\Phi^+}}  % Bell state Phi+
\newcommand{\bellphim}{\ket{\Phi^-}} % Bell state Phi-
\newcommand{\bellpsi}{\ket{\Psi^+}}  % Bell state Psi+
\newcommand{\bellpsim}{\ket{\Psi^-}} % Bell state Psi-

% D. Other notation:
\newcommand{\expectation}[1]{\left\langle {#1} \right\rangle }
\newcommand{\krondelta}[1]{\delta_{#1}}