\documentclass[12pt]{article}

% Geometry
\usepackage[legalpaper, portrait, margin=1in]{geometry}

% Math packages
\usepackage{amsmath}
\usepackage{amssymb}
\usepackage{amsthm}
\usepackage{mathtools}
\usepackage{physics}
% \usepackage{braket}

% Graphics and drawing
\usepackage{graphicx}
\usepackage{tikz}
\usepackage{quantikz} % For quantum circuit drawing

% Code formatting
\usepackage{minted}

% Layout and formatting
\usepackage{titling}
\usepackage{fancyhdr}
\usepackage{titlesec}
\usepackage{multicol}
\usepackage{xcolor}

% Tables
\usepackage{multirow}
\usepackage{array}
\usepackage{booktabs}
\usepackage{tabularx}


\usemintedstyle{manni}
\graphicspath{ {./figures/} }
\newcommand{\Rlogo}{\protect\includegraphics[height=1.8ex,keepaspectratio]{Rlogo.png}}

% config.tex
\newcommand{\COURSECODE}{QCI402}
\newcommand{\FULLCOURSENAME}{Mathmatical Foundations for Quantum Computing}
\newcommand{\PROFESSORNAME}{Dr. Miao Yu}
\newcommand{\STUDENTNAME}{Mingjia Guan}
\newcommand{\SEMESTER}{Fall}
\newcommand{\YEAR}{2025}
\newcommand{\UNIVERSITY}{Northern University}

%common commands
% math symbols 
\newcommand{\R}{\mathbb{R}}
\newcommand{\C}{\mathbb{C}}
\newcommand{\N}{\mathbb{N}}
\newcommand{\Z}{\mathbb{Z}}
\newcommand{\Q}{\mathbb{Q}}

\newcommand{\Rn}{\mathbb{R}^n}
\newcommand{\Cn}{\mathbb{C}^n}
\newcommand{\Nn}{\mathbb{N}^n}
\newcommand{\Zn}{\mathbb{Z}^n}
\newcommand{\Qn}{\mathbb{Q}^n}

\newcommand{\conj}[1]{{#1}^*}

\newcommand{\od}[2]{\frac{d#1}{d#2}}
\newcommand{\pd}[2]{\frac{\partial#1}{\partial#2}}

\newcommand{\zerovec}{\mathbf{0}}
\newcommand{\vvec}{\mathbf{v}}
\newcommand{\uvec}{\mathbf{u}}
\newcommand{\wvec}{\mathbf{w}}

\newcommand{\ibase}{\mathbf{\hat{i}}}
\newcommand{\jbase}{\mathbf{\hat{j}}}
\newcommand{\kbase}{\mathbf{\hat{k}}}


% Basic quantum states
\newcommand{\ketphi}{\ket{\phi}}
\newcommand{\braphi}{\bra{\phi}}
\newcommand{\ketpsi}{\ket{\psi}}
\newcommand{\brapsi}{\bra{\psi}}

% Common quantum states
\newcommand{\qzero}{\ket{0}}
\newcommand{\qone}{\ket{1}}
\newcommand{\qplus}{\ket{+}}
\newcommand{\qminus}{\ket{-}}
\newcommand{\qpsi}{\ket{\psi}}
\newcommand{\qphi}{\ket{\phi}}
\newcommand{\qchi}{\ket{\chi}}

% Bell states
\newcommand{\bellphi}{\ket{\Phi^+}}
\newcommand{\bellphim}{\ket{\Phi^-}}
\newcommand{\bellpsi}{\ket{\Psi^+}}
\newcommand{\bellpsim}{\ket{\Psi^-}}

% \documentclass[12pt]{article}

\usepackage[legalpaper, portrait, margin=0.75in]{geometry}

\usepackage{amsmath}
\usepackage{amssymb}
\usepackage{amsthm}
\usepackage{physics}
\usepackage{mathtools}
\usepackage{braket}
\usepackage{tikz}
\usepackage{quantikz}
\usepackage{multicol}
\usepackage{xcolor}
\usepackage{fancyhdr}
\usepackage{titlesec}
\usepackage{multirow}
\usepackage{array}
\usepackage{booktabs}
\usepackage{quantikz}
\usepackage{tabularx}


% Page setup
\pagestyle{fancy}
\fancyhf{}
\fancyhead[L]{Quantum Computing LaTeX Quick Reference}
\fancyhead[R]{\thepage}
\setlength{\headheight}{14.5pt}

% Title formatting
\titleformat{\section}{\large\bfseries\color{blue!70!black}}{\thesection}{1em}{}
\titleformat{\subsection}{\normalsize\bfseries\color{blue!50!black}}{\thesubsection}{1em}{}

\title{\textbf{\Large Quantum Computing LaTeX Quick Reference}}
\author{Quick Command Sheet for Quantum Computing Lecture Notes}
\date{}

% =============================================================================
% QUANTUM STATE NOTATION
% =============================================================================

% Basic quantum states
% \newcommand{\ket}{|#1\rangle}
% \newcommand{\bra}{\langle#1|}
% \newcommand{\braket}{\langle#1|#2\rangle}
% \newcommand{\ketbra}{|#1\rangle\langle#2|}
\newcommand{\expectation}[1]{\langle#1\rangle}

\newcommand{\ketphi}{\ket{\phi}}
\newcommand{\braphi}{\bra{\phi}}
\newcommand{\ketpsi}{\ket{\psi}}
\newcommand{\brapsi}{\bra{\psi}}

% Common quantum states
\newcommand{\qzero}{\ket{0}}
\newcommand{\qone}{\ket{1}}
\newcommand{\qplus}{\ket{+}}
\newcommand{\qminus}{\ket{-}}
\newcommand{\qpsi}{\ket{\psi}}
\newcommand{\qphi}{\ket{\phi}}
\newcommand{\qchi}{\ket{\chi}}

% Bell states
\newcommand{\bellphi}{\ket{\Phi^+}}
\newcommand{\bellphim}{\ket{\Phi^-}}
\newcommand{\bellpsi}{\ket{\Psi^+}}
\newcommand{\bellpsim}{\ket{\Psi^-}}

% GHZ and W states
\newcommand{\GHZ}{\ket{\text{GHZ}}}
\newcommand{\Wstate}{\ket{\text{W}}}

% =============================================================================
% QUANTUM OPERATORS AND GATES
% =============================================================================

% Pauli matrices
\newcommand{\sx}{\sigma_x}
\newcommand{\sy}{\sigma_y}
\newcommand{\sz}{\sigma_z}
\newcommand{\si}{\sigma_0}
\newcommand{\pauliX}{\sigma_X}
\newcommand{\pauliY}{\sigma_Y}
\newcommand{\pauliZ}{\sigma_Z}

% Common quantum gates
\newcommand{\hadamard}{H}
\newcommand{\cnot}{\text{CNOT}}
\newcommand{\ccnot}{\text{CCNOT}}
\newcommand{\toffoli}{\text{Toffoli}}
\newcommand{\fredkin}{\text{Fredkin}}
% \newcommand{\swap}{\text{SWAP}}

% Rotation gates
\newcommand{\rx}[1]{R_x(#1)}
\newcommand{\ry}[1]{R_y(#1)}
\newcommand{\rz}[1]{R_z(#1)}
\newcommand{\rn}[2]{R_{\hat{n}}(#1, #2)}

% Phase gates
\newcommand{\phaseS}{S}
\newcommand{\phaseT}{T}
\newcommand{\phaseP}[1]{P(#1)}

% Measurement
% \newcommand{\measure}{\mathcal{M}}
% \newcommand{\proj}[1]{P_{#1}}
\newcommand{\projzero}{\proj{0}}
\newcommand{\projone}{\proj{1}}

% =============================================================================
% QUANTUM OPERATIONS AND PROCESSES
% =============================================================================

% Unitary operations
\newcommand{\unitary}{\mathcal{U}}
\newcommand{\unitarydagger}{\mathcal{U}^\dagger}

% Quantum channels
\newcommand{\channel}{\mathcal{E}}
\newcommand{\noise}{\mathcal{N}}
\newcommand{\depolar}{\mathcal{D}}
\newcommand{\bitflip}{\mathcal{B}}
\newcommand{\phaseflip}{\mathcal{P}}

% Tensor products
\newcommand{\tensor}{\otimes}
% \newcommand{\bigotimes}{\bigotimes}

% Trace operations
% \newcommand{\tr}{\text{Tr}}
\newcommand{\ptr}[1]{\text{Tr}_{#1}}

% =============================================================================
% QUANTUM INFORMATION MEASURES
% =============================================================================

% Entropies
\newcommand{\entropy}[1]{H(#1)}
\newcommand{\condentropy}[2]{H(#1|#2)}
\newcommand{\mutualinfo}[2]{I(#1:#2)}
\newcommand{\vonneumann}[1]{S(#1)}

% Distances and fidelities
\newcommand{\fidelity}[2]{F(#1, #2)}
\newcommand{\tracedist}[2]{D(#1, #2)}

% =============================================================================
% QUANTUM ALGORITHMS
% =============================================================================

% Algorithm names
\newcommand{\QFT}{\text{QFT}}
\newcommand{\iQFT}{\text{QFT}^{-1}}
\newcommand{\grover}{\text{Grover}}
\newcommand{\shor}{\text{Shor}}

% Oracle notation
\newcommand{\oracle}{O}
\newcommand{\oraclef}{O_f}
\newcommand{\oracleg}{O_g}

% Amplitudes
\newcommand{\amp}[1]{\alpha_{#1}}
\newcommand{\betamp}[1]{\beta_{#1}}

% =============================================================================
% QUANTUM ERROR CORRECTION
% =============================================================================

% Stabilizers
\newcommand{\stab}{\mathcal{S}}
\newcommand{\stabgen}{g}
\newcommand{\syndrome}{s}

% Error operators
\newcommand{\error}{E}
\newcommand{\errorX}{X}
\newcommand{\errorY}{Y}
\newcommand{\errorZ}{Z}

% Code spaces
\newcommand{\codespace}{\mathcal{C}}
\newcommand{\codeword}{\ket{\overline{\psi}}}

% =============================================================================
% QUANTUM COMPLEXITY
% =============================================================================

% Complexity classes
\newcommand{\BQP}{\text{BQP}}
\newcommand{\QMA}{\text{QMA}}
\newcommand{\QPSPACE}{\text{QPSPACE}}

% =============================================================================
% QUANTUM CIRCUITS AND DIAGRAMS
% =============================================================================

% Circuit elements (for use with quantikz)
% \newcommand{\qw}{\qw}
% \newcommand{\gate}[1]{\gate{#1}}
% \newcommand{\ctrl}{\ctrl{1}}
% \newcommand{\targ}{\targ{}}
% \newcommand{\meter}{\meter{}}

% =============================================================================
% MATHEMATICAL UTILITIES
% =============================================================================

% Complex numbers
\newcommand{\ii}{\mathrm{i}}
% \newcommand{\real}{\text{Re}}
\newcommand{\imag}{\text{Im}}
\newcommand{\conj}[1]{#1^*}

% Matrix operations
% \newcommand{\rank}{\text{rank}}
\newcommand{\nullspace}{\text{null}}
\newcommand{\colspace}{\text{col}}
\newcommand{\rowspace}{\text{row}}

% Probability
\newcommand{\prob}[1]{\text{Pr}[#1]}
\newcommand{\expect}[1]{\mathbb{E}[#1]}
% \newcommand{\var}[1]{\text{Var}[#1]}

% Sets
\newcommand{\hilbert}{\mathcal{H}}
\newcommand{\density}{\mathcal{D}}
\newcommand{\positive}{\mathcal{P}}

% =============================================================================
% QUANTUM PHYSICS NOTATION
% =============================================================================

% Time evolution
\newcommand{\hamiltonian}{\mathcal{H}}
\newcommand{\timeevol}[1]{e^{-\ii #1 t/\hbar}}
% \newcommand{\hbar}{\hbar}

% Angular momentum
\newcommand{\angmom}{\mathbf{J}}
\newcommand{\spin}{\mathbf{S}}
\newcommand{\orbital}{\mathbf{L}}

% Commutators and anticommutators
% \newcommand{\comm}[2]{[#1, #2]}
\newcommand{\anticomm}[2]{\{#1, #2\}}

% =============================================================================
% ENVIRONMENTS FOR QUANTUM COMPUTING
% =============================================================================

% Theorem-like environments (already defined in main template)
% Additional quantum-specific environments
\theoremstyle{definition}
\newtheorem{algorithm}{Algorithm}[section]
\newtheorem{protocol}{Protocol}[section]
\newtheorem{circuit}{Circuit}[section]

\theoremstyle{remark}
\newtheorem{quantum-note}{Quantum Note}[section]
\newtheorem{classical-note}{Classical Note}[section]

\begin{document}
\maketitle
\thispagestyle{fancy}

\begin{multicols}{2}

\section{Basic Quantum State Notation}

\subsection{State Vectors}
\begin{alignat}{2}
\text{Ket:}          &\quad && \ket{\psi} \text{ or } \qpsi \\
\text{Bra:}          &\quad && \bra{\psi} \\
\text{Braket:}       &\quad && \braket{\psi | \phi} \\
\text{Outer product:}&\quad && \ketbra{\psi}{\phi} \\
\text{Expectation:}  &\quad && \expectation{A} = \braket{\psi}{A|\psi}
\end{alignat}

\subsection{Common States}
\begin{alignat}{2}
\text{Computational basis:} &\quad && \qzero, \qone \\
\text{Superposition:}       &\quad && \qplus = \frac{1}{\sqrt{2}}(\qzero + \qone) \\
                             &\quad && \qminus = \frac{1}{\sqrt{2}}(\qzero - \qone) \\
\text{Bell states:}         &\quad && \bellphi, \bellphim, \bellpsi, \bellpsim \\
\text{Multi-qubit:}         &\quad && \GHZ, \Wstate
\end{alignat}


\section{Quantum Gates and Operations}

\subsection{Single-Qubit Gates}
\begin{alignat}{2}
\text{Pauli matrices:} &\quad && \pauliX, \pauliY, \pauliZ \\
\text{Hadamard:}       &\quad && \hadamard \\
\text{Phase gates:}    &\quad && \phaseS, \phaseT, \phaseP{\theta} \\
\text{Rotations:}      &\quad && \rx{\theta}, \ry{\theta}, \rz{\theta}
\end{alignat}

\subsection{Multi-Qubit Gates}
\begin{alignat}{2}
\text{CNOT:}     &\quad && \cnot \\
\text{Toffoli:}  &\quad && \toffoli \text{ or } \ccnot \\
\text{Fredkin:}  &\quad && \fredkin \\
\text{SWAP:}     &\quad && \text{SAWP}
\end{alignat}

\section{Quantum Measurements}

\begin{alignat}{2}
\text{Measurement operator:} &\quad && \mathcal{M} \\
\text{Projectors:}           &\quad && \projzero, \projone, \P_{i} \\
\text{Probability:}          &\quad && \prob{i} = \braket{\psi}{\proj{i}|\psi}
\end{alignat}

% \newcommand{\measure}{\mathcal{M}}
% \newcommand{\proj}[1]{P_{#1}}


\section{Quantum Channels and Noise}

\begin{alignat}{2}
\text{General channel:} &\quad && \channel(\rho) \\
\text{Depolarizing:}    &\quad && \depolar_p(\rho) \\
\text{Bit flip:}        &\quad && \bitflip_p(\rho) \\
\text{Phase flip:}      &\quad && \phaseflip_p(\rho)
\end{alignat}


\section{Quantum Information Measures}

\begin{alignat}{2}
\text{Von Neumann entropy:} &\quad && \vonneumann{\rho} \\
\text{Conditional entropy:} &\quad && \condentropy{A}{B} \\
\text{Mutual information:}  &\quad && \mutualinfo{A}{B} \\
\text{Fidelity:}             &\quad && \fidelity{\rho}{\sigma} \\
\text{Trace distance:}       &\quad && \tracedist{\rho}{\sigma}
\end{alignat}


\section{Quantum Algorithms}

\subsection{Fourier Transform}
\begin{alignat}{2}
\text{QFT:}         &\quad && \QFT\ket{x} = \frac{1}{\sqrt{N}} \sum_{k=0}^{N-1} e^{2\pi i xk/N}\ket{k} \\
\text{Inverse QFT:} &\quad && \iQFT
\end{alignat}

\subsection{Oracles}
\begin{alignat}{2}
\text{Boolean oracle:} &\quad && \oraclef\ket{x}\ket{y} = \ket{x}\ket{y \oplus f(x)} \\
\text{Phase oracle:}   &\quad && \oraclef\ket{x} = (-1)^{f(x)}\ket{x}
\end{alignat}

\section{Error Correction}

\begin{alignat}{2}
\text{Stabilizer group:}  &\quad && \stab = \langle g_1, g_2, \ldots, g_k \rangle \\
\text{Syndrome:}          &\quad && \syndrome = (s_1, s_2, \ldots, s_k) \\
\text{Code space:}        &\quad && \codespace \\
\text{Logical codeword:}  &\quad && \codeword
\end{alignat}


\section{Mathematical Utilities}

\subsection{Complex Numbers}
\begin{alignat}{2}
\text{Imaginary unit:}     &\quad && \ii \\
\text{Real part:}          &\quad && \real(z) \\
\text{Imaginary part:}     &\quad && \imag(z) \\
\text{Complex conjugate:}  &\quad && \conj{z}
\end{alignat}

\subsection{Linear Algebra}
\begin{alignat}{2}
\text{Tensor product:}  &\quad && A \tensor B \\
\text{Trace:}           &\quad && \tr(A) \\
\text{Partial trace:}   &\quad && \ptr{B}(\rho_{AB}) \\
\text{Rank:}            &\quad && \rank(A)
\end{alignat}

\subsection{Probability}
\begin{alignat}{2}
\text{Probability:}  &\quad && \prob{\text{event}} \\
\text{Expectation:}  &\quad && \expect{X} \\
\text{Variance:}     &\quad && \var{X}
\end{alignat}

\end{multicols}


\section{Circuit Notation (with quantikz)}

Basic circuit elements:
\begin{verbatim}
\begin{quantikz}
\lstick{$\ket{0}$} & \gate{H} & \ctrl{1} & \meter{} \\
\lstick{$\ket{0}$} & \qw & \targ{} & \qw
\end{quantikz}
\end{verbatim}

\section{Complexity Classes}

\begin{alignat}{2}
\text{Bounded-error quantum polynomial:} &\quad && \BQP \\
\text{Quantum Merlin-Arthur:}           &\quad && \QMA \\
\text{Quantum polynomial space:}        &\quad && \QPSPACE
\end{alignat}


\section{Usage Examples}

\subsection{Quantum Teleportation Protocol}
The quantum teleportation protocol transfers the state $\qpsi = \alpha\qzero + \beta\qone$ using the following steps:

\begin{protocol}
\begin{enumerate}
\item Alice and Bob share the Bell state $\bellphi = \frac{1}{\sqrt{2}}(\ket{00} + \ket{11})$
\item Alice performs a Bell measurement on her qubit and the qubit to be teleported
\item Alice sends the classical result to Bob
\item Bob applies the appropriate correction based on Alice's measurement
\end{enumerate}
\end{protocol}

\subsection{Grover's Algorithm}
Grover's algorithm amplifies the amplitude of marked states:

\begin{algorithm}
\begin{enumerate}
\item Initialize: $\ket{\psi_0} = \frac{1}{\sqrt{N}}\sum_{x=0}^{N-1}\ket{x}$
\item Apply $G = -\hadamard^{\tensor n}\proj{0}^{\tensor n}\hadamard^{\tensor n}\oraclef$ for $O(\sqrt{N})$ iterations
\item Measure to obtain the marked state with high probability
\end{enumerate}
\end{algorithm}

\subsection{Quantum Error Correction Example}
The 3-qubit bit flip code protects against single bit flip errors:

\begin{circuit}
Encoding: $\ket{0} \rightarrow \ket{000}$, $\ket{1} \rightarrow \ket{111}$

Syndrome measurement: $s_1 = Z_1Z_2$, $s_2 = Z_2Z_3$
\end{circuit}

\section{Common Quantum Gates}

\begin{table}[h!]
\centering
\renewcommand{\arraystretch}{1.6}
\begin{tabularx}{\textwidth}{X X}
% --- Left column: Single-Qubit ---
\begin{tabular}{cl}
\toprule
\textbf{Gate} & \textbf{Matrix} \\
\midrule
Hadamard ($H$) &
$\dfrac{1}{\sqrt{2}} \begin{bmatrix}
1 & 1 \\
1 & -1
\end{bmatrix}$ \\
Identity ($I$) &
$\begin{bmatrix}
1 & 0 \\
0 & 1
\end{bmatrix}$ \\
Phase ($S$) &
$\begin{bmatrix}
1 & 0 \\
0 & i
\end{bmatrix}$ \\
T-gate ($T$) &
$\begin{bmatrix}
1 & 0 \\
0 & e^{i\pi/4}
\end{bmatrix}$ \\
NOT ($X$) &
$\begin{bmatrix}
0 & 1 \\
1 & 0
\end{bmatrix}$ \\
Y-gate ($Y$) &
$\begin{bmatrix}
0 & -i \\
i & 0
\end{bmatrix}$ \\
Z-gate ($Z$) &
$\begin{bmatrix}
1 & 0 \\
0 & -1
\end{bmatrix}$ \\
Rotation $R(\theta)$ / Phase $P(\theta)$ &
$\begin{bmatrix}
1 & 0 \\
0 & e^{i\theta}
\end{bmatrix}$ \\
\bottomrule
\end{tabular}
&
% --- Right column: Multi-Qubit ---
\begin{tabular}{cl}
\toprule
\textbf{Gate} & \textbf{Matrix} \\
\midrule
CNOT ($CX$) &
$\begin{bmatrix}
1 & 0 & 0 & 0 \\
0 & 1 & 0 & 0 \\
0 & 0 & 0 & 1 \\
0 & 0 & 1 & 0
\end{bmatrix}$ \\
Controlled-Z ($CZ$) &
$\begin{bmatrix}
1 & 0 & 0 & 0 \\
0 & 1 & 0 & 0 \\
0 & 0 & 1 & 0 \\
0 & 0 & 0 & -1
\end{bmatrix}$ \\
Controlled-$U$ ($CU$) &
$\begin{bmatrix}
1 & 0 & 0 & 0 \\
0 & 1 & 0 & 0 \\
0 & 0 & U_{00} & U_{01} \\
0 & 0 & U_{10} & U_{11}
\end{bmatrix}$ \\
SWAP &
$\begin{bmatrix}
1 & 0 & 0 & 0 \\
0 & 0 & 1 & 0 \\
0 & 1 & 0 & 0 \\
0 & 0 & 0 & 1
\end{bmatrix}$ \\
Toffoli (CCNOT) &
\scriptsize
$\begin{bmatrix}
1 & 0 & 0 & 0 & 0 & 0 & 0 & 0 \\
0 & 1 & 0 & 0 & 0 & 0 & 0 & 0 \\
0 & 0 & 1 & 0 & 0 & 0 & 0 & 0 \\
0 & 0 & 0 & 1 & 0 & 0 & 0 & 0 \\
0 & 0 & 0 & 0 & 1 & 0 & 0 & 0 \\
0 & 0 & 0 & 0 & 0 & 1 & 0 & 0 \\
0 & 0 & 0 & 0 & 0 & 0 & 0 & 1 \\
0 & 0 & 0 & 0 & 0 & 0 & 1 & 0
\end{bmatrix}$
\normalsize \\
\bottomrule
\end{tabular}
\end{tabularx}
\end{table}




\begin{figure}[h]
\centering
\begin{quantikz}
\lstick{$\ket{0}$} & \gate{H} & \ctrl{1} & \meter{} \\
\lstick{$\ket{0}$} & \qw    & \targ{}  & \meter{}
\end{quantikz}
\caption{Quantum circuit for preparing and measuring a Bell state}
\end{figure}



\end{document}

\theoremstyle{plain}
\newtheorem{theorem}{Theorem}[section]
\newtheorem{definition}[theorem]{Definition}
\newtheorem{postulate}[theorem]{Postulate}
\newtheorem{corollary}[theorem]{Corollary}
\newtheorem{lemma}[theorem]{Lemma}

\newtheorem*{example}{Example}

\theoremstyle{definition}
\newtheorem*{solution}{Solution}

\title{\COURSECODE\ - \FULLCOURSENAME}
\author{\PROFESSORNAME\ - \STUDENTNAME}
\date{\SEMESTER\ Semester \YEAR}

\begin{document}

\maketitle

\hfill

The underpinnings of all scientific advancements is the ability to express natural phenomena with the art of Mathmatics; this is no different for the subject of Quantum Computing. While the boundaries of quantum computing have been pushed beyond limits in theoretical terms on university blackboards, it has become of great interest to realize the theoretical computational power with the advances of hardware and technology. 

However, these notes mainly concerns itself with the mathmatical underpinnings of quantum computing that the course surrounds itself with. \FULLCOURSENAME takes a scaffolding approach designed to efficiently convey the required theoretical understanding of mathmatics in order to able to learn quantum computing. As of writing, we are basing the notes on verison one of the textbook published in March 2025. In this text, we will primarily be using dirac notation for the expression of vectors, operators, and their interactions. 

\tableofcontents

\break


\section{Summation and Product Notations}

This section primarily focuses on the common notations applied across mathmatics to denote and shorten addition and product notation. 

\subsection{Summation over a single Variable}

The sigma notation is defined as follows

$$\sum_{i=1}^{n}f(i)$$

where we use sigma $\sum$ to represent the sum of a series. For example, the sum of all numbers in a series beginning with $m$ and ending at index $n$ is written as:

$$\sum_{i=m}^{n} a_i = a_m + a_{m+1} + a_{m+2} + \cdots + a_{n-1} + a_n$$

Sums can also be infinite, commonly seen when Sigma looks as follows: $\sum^{\infty}_{i=m}$. Infinite sums are either convergent or divergent. A few of the most common converging infinite sums are as follows:

$$\sum_{i=0}^{\infty} \frac{1}{2^i} = 1 + \frac{1}{2} + \frac{1}{4} + \cdots = 2$$
$$\sum_{i=0}^{\infty}  \frac{1}{i^2} = \frac{1}{1^2} + \frac{1}{2^2} + \frac{1}{3^2} + \cdots = \frac{\pi^2}{6}$$

The first example is an infinite geometric series, and the sum of the first $n$ terms is given by:

$$S_n = \sum_{i=0}^{n} \frac{1}{2^i} = \frac{1 - \frac{1}{2^n}}{1 - \frac{1}{2}}$$

As $n \rightarrow \infty, \frac{1}{2^n} \rightarrow 0$. Consequently, $S_n \rightarrow \frac{1}{1 - \frac{1}{2}} = 2$. A rigorous proof of the second example requires extensive calculus and is not immediately obvious. While any mathmatical symbol can be used for the index of a summation, it is more practical to use something other than $i$ as in the context of complex numbers, $i$ commonly denotes the complex number$\sqrt{-1}$. moreover, sume can also be specified using descriptions. For example, 

$$\sum_{p \in P} f(p) \qquad P \in \mathbb{N^\prime}$$

where $\mathbb{N^\prime}$ is the set of all prime numbers. Summations can also contain parameters other than the index, which results in functions of those parameters. For example the discrete Fourier transform (DFT) is given by

$$\tilde{x}_k = \frac{1}{\sqrt{N}} \sum_{n=0}^{N - 1} x_n e^{- \frac{2 \pi i}{N} kn}, \quad k = 0, 1, \cdots N-1$$

where $x_n$ represents the $N$ values index by $n$ and $\tilde{x}_k$ are the Fourier coefficients. Here, $i$ is the imaginary numebr and $N$ is a positive integer representing the dimension fo the DFT, of which we will cover in greater depth in Chapter 3. The following are some useful summation forumae commonly encountered in quantum computing:

\[
\sum_{i=1}^{n} i = \frac{n(n+1)}{2}
\]

\[
\sum_{i=1}^{n} i^2 = \frac{n(n+1)(2n+1)}{6}
\]

\[
\sum_{i=1}^{n} i^3 = \left(\frac{n(n+1)}{2}\right)^2
\]

\[
\sum_{i=0}^{n} \left(a_0 + id\right) = (n+1)\left(a_0 + \frac{nd}{2}\right) \quad \text{(arithmetic series)}
\]

\[
\sum_{i=0}^{n} a^i = \frac{1 - a^{n+1}}{1 - a} \quad \text{(geometric series)}
\]

\[
(a + b)^n = \sum_{i=0}^{n} \binom{n}{i} a^{n-i} b^i \quad \text{(binomial theorem)}
\]

\[
\frac{1}{1 - x} = \sum_{n=0}^{\infty} x^n = 1 + x + x^2 + x^3 + \cdots \quad (\lvert x \rvert < 1)
\]

\[
\frac{1}{(1-x)^2} = \sum_{n=1}^{\infty} n x^{n-1} = 1 + 2x + 3x^2 + 4x^3 + \cdots \quad (\lvert x \rvert < 1)
\]

\[
\ln(1 + x) = \sum_{n=1}^{\infty} \frac{(-1)^{n+1}}{n} x^n = x - \frac{x^2}{2} + \frac{x^3}{3} - \cdots \quad (\lvert x \rvert < 1)
\]

\[
e^x = \sum_{n=0}^{\infty} \frac{x^n}{n!} = 1 + x + \frac{x^2}{2!} + \frac{x^3}{3!} + \cdots
\]

\[
\sin x = \sum_{n=0}^{\infty} \frac{(-1)^n}{(2n+1)!} x^{2n+1} = x - \frac{x^3}{3!} + \frac{x^5}{5!} - \cdots
\]

\[
\cos x = \sum_{n=0}^{\infty} \frac{(-1)^n}{(2n)!} x^{2n} = 1 - \frac{x^2}{2!} + \frac{x^4}{4!} - \cdots
\]

Below are also a list of the common summations rules and manupulations:

\[
\sum_{i=m}^{n} a_i = \sum_{j=m}^{n} a_j \quad \text{(change of index variable)}
\]

\[
\sum_{i=s}^{t} f(i) = \sum_{n=s}^{t} f(n) \quad \text{(change of index variable)}
\]

\[
\sum_{n=s}^{t} f(n) = \sum_{n=s}^{j} f(n) + \sum_{n=j+1}^{t} f(n) \quad \text{(splitting a sum)}
\]

\[
\sum_{n=s}^{t} f(n) = \sum_{n=0}^{t-s} f(t-n) \quad \text{(reverse order)}
\]

\[
\sum_{n=s}^{t} f(n) = \sum_{n=s+p}^{t+p} f(n-p) \quad \text{(index shift)}
\]

\[
\sum_{n=s}^{t} a \cdot f(n) = a \cdot \sum_{n=s}^{t} f(n) \quad \text{(distributivity)}
\]

\[
\sum_{n=s}^{t} f(n) \pm \sum_{n=s}^{t} g(n) = \sum_{n=s}^{t} \left(f(n) \pm g(n)\right) \quad \text{(commutativity)}
\]


\subsection{Products and other Notations}

Similar to the $\sum$ notation for addition, the $\prod$ (Pi) symbol is also more commonly used to dentoe the product of a series of terms. In this 

$$\prod_{i=m}^{n} a_i = a_m \cdot a_{m+1} \cdot a_{m+2} \cdot \cdots \cdot a_{n-1} \cdot a_n$$ 

for example, the factorial of $n$ is expressed as 

$$\prod_{i=0}^{n} i = n!$$

and the relationship between $\sum$ and $\prod$, which are 

$$b^{\sum_{n=s}^{t} f(n)} = \prod_{n=s}^{t} b^{f(n)}$$

$$\sum_{n=s}^{t} \log_b f(n) = log_b \prod_{n=s}^{t} f(n)$$

It is worth noting that in quantum computing and linear algebra, there are a few special notations such as the modulo-2 sum (bitwise XOR), or in other contexts the direct sum of linear spaces, represented by $\oplus$, and the tensor product represented by $\otimes$.

\subsection{Summation over Multiple Variables}

The double summation over a rectangular array is given by 

\begin{align*}
    \sum_{i=1,j=1}^{n_1, n_2} a_{i,j} &= \sum_{i=1}^{n_1} \sum_{j=1}^{n_2} a_{i, j} = \sum_{j=1}^{n_2} \sum_{i=1}^{n_1} a_{i, j} \\
    &= a_{1, 1} + a_{1, 2} + a_{1, 3} + a_{1, 4} + \cdots + a_{1, n_2} \\
    &+ a_{2, 1} + a_{2, 2} + a_{2, 3} + a_{2, 4} + \cdots + a_{2, n_2} \\
    & + a_{3, 1} + a_{3, 2} + a_{3, 3} + a_{3, 4} + \cdots + a_{3, n_2} \\
    & + a_{4, 1} + a_{4, 2} + a_{4, 3} + a_{4, 4} + \cdots + a_{4, n_2} \\
    & + \cdots \\
    & + a_{n_1, 1} + a_{n_1, 2} + a_{n_1, 3} + a_{n_1, 4} + \cdots + a_{n_1, n_2}
\end{align*}

Here, $\sum_{i=1}^{n_1} \sum_{j=1}^{n_2}$ represents summing over each row first and then summing the results, while $\sum_{j=1}^{n_2} \sum_{i=1}^{n_1}$ will represent summing over the columns and then summing those results. The term $\sum_{i=1,j=1}^{n_1, n_2} a_{i,j}$ represents the summation over the rectangular array, irrespecive of the order. The product of two sums can be expanded into a double sum as follows:

\begin{align*}
\left( \sum_{i=1}^{m} a_i \right) \left( \sum_{j=1}^{n} b_j \right)
    &= (a_1 + a_2 + \cdots + a_m)(b_1 + b_2 + \cdots + b_n) \\
    &=\; a_1b_1 + a_1b_2 + a_1b_3 + a_1b_4 + \cdots + a_1b_n \\
    &+ a_2b_1 + a_2b_2 + a_2b_3 + a_2b_4 + \cdots + a_2b_n \\
    &+ a_3b_1 + a_3b_2 + a_3b_3 + a_3b_4 + \cdots + a_3b_n \\
    &+ \cdots \\
    &+ a_mb_1 + a_mb_2 + a_mb_3 + a_mb_4 + \cdots + a_mb_n \\
    &= \sum_{i=1}^{m} \sum_{j=1}^{n} a_i b_j = \sum_{i=1}^{m} a_i \sum_{j=1}^{n} b_j
\end{align*}

which is actually rather intuitive given how the expansion of the standard expansion of the term $(a+b)^2$ plays out, a more elementary application of the distributive property which the above equation generalizes over. For a triangular matrix, in this case the lower triangular matrix, the sum is given by

\begin{align*}
    \sum_{1 \leq j \leq n}^{} a_{i,j} &= \sum_{i=1}^{n} \sum_{j=1}^{i} a_{i, j} = 
    \sum_{j=1}^{n} \sum_{i=j}^{n} a_{i, j} = 
    \sum_{j=0}^{n-1} \sum_{j=1}^{n-j} a_{i+j, i} \\
    &= a_{1, 1} \\
    & + a_{2, 1} + a_{2, 2}  \\
    & + a_{3, 1} + a_{3, 2} + a_{3, 3} \\
    & + a_{4, 1} + a_{4, 2} + a_{4, 3} + a_{4, 4} \\
    & + \cdots \\
    & + a_{n, 1} + a_{n, 2} + a_{n, 3} + a_{n, 4} + \cdots + a_{n, n}
\end{align*}

where the term $\sum_{1 \leq j \leq n}^{} a_{i,j}$ denotes the summation over all elements in a lower triangular array including the diagonal. The first notation variation will sum up each row to the $i$th element then aggregate while the second notation sums each column starting from the $j$th element downwards then aggregate the sums. The final expression will sum along the diagonal where $j=0$ represents the main diagonal and $j=n-1$ is the first off-diagonal, which is a single term. 

\begin{example}
    Say we would like to expand the product of $\left(1 + x_i\right)$ from $1$ to $n$. We have 

    $$\prod_{i=1}^{n} \left(1 + x_i\right) = 1 + \sum_{k=1}^{n} \left(\sum_{1 \leq i_1 < \cdots < i_k \leq n} \prod_{j=1}^{k} x_{i_j}\right)$$
\end{example}

This formula represents the \textit{multinomial expansion} of a product. When you expand the equation by hand, you get the product

$$\prod_{i=1}^{n} \left(1 + x_i\right) = (1 + x_1)(1 + x_2)\cdots(1 + x_n)$$

If we break this down, we see that the outer summation $\sum_{k=1}^{n}$ will go through each possible summation size in terms of the variables in question, and that the inner summation $\sum_{1 \leq i_1 < \cdots < i_k \leq n}$ will iterate through each possible unique product of the variables. while ensuring that they are unique. Not sure how this works, but if all $x_i$ are the same, then wesee that the equation actually simplifies to a subset of the binomial theorem 

$$(1 + x)^n = \sum_{k=0}^{n} \binom{n}{k} x^k$$

where $\binom{n}{k}$ is the binomial coefficient representing the numebr of ways to choose $k$ elements from a set of $n$ distinct elements.









\end{document}