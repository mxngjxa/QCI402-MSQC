\documentclass[12pt]{article}

% Geometry
\usepackage[legalpaper, portrait, margin=1in]{geometry}

% Math packages
\usepackage{amsmath}
\usepackage{amssymb}
\usepackage{amsthm}
\usepackage{mathtools}
\usepackage{physics}
% \usepackage{braket}

% Graphics and drawing
\usepackage{graphicx}
\usepackage{tikz}
\usepackage{quantikz} % For quantum circuit drawing

% Code formatting
\usepackage{minted}

% Layout and formatting
\usepackage{titling}
\usepackage{fancyhdr}
\usepackage{titlesec}
\usepackage{multicol}
\usepackage{xcolor}

% Tables
\usepackage{multirow}
\usepackage{array}
\usepackage{booktabs}
\usepackage{tabularx}


\usemintedstyle{manni}
\graphicspath{ {./figures/} }
\newcommand{\Rlogo}{\protect\includegraphics[height=1.8ex,keepaspectratio]{Rlogo.png}}

% config.tex
\newcommand{\COURSECODE}{QCI402}
\newcommand{\FULLCOURSENAME}{Mathmatical Foundations for Quantum Computing}
\newcommand{\PROFESSORNAME}{Dr. Miao Yu}
\newcommand{\STUDENTNAME}{Mingjia Guan}
\newcommand{\SEMESTER}{Fall}
\newcommand{\YEAR}{2025}
\newcommand{\UNIVERSITY}{Northern University}

%common commands
% math symbols 
\newcommand{\R}{\mathbb{R}}
\newcommand{\C}{\mathbb{C}}
\newcommand{\N}{\mathbb{N}}
\newcommand{\Z}{\mathbb{Z}}
\newcommand{\Q}{\mathbb{Q}}

\newcommand{\Rn}{\mathbb{R}^n}
\newcommand{\Cn}{\mathbb{C}^n}
\newcommand{\Nn}{\mathbb{N}^n}
\newcommand{\Zn}{\mathbb{Z}^n}
\newcommand{\Qn}{\mathbb{Q}^n}

\newcommand{\conj}[1]{{#1}^*}

\newcommand{\od}[2]{\frac{d#1}{d#2}}
\newcommand{\pd}[2]{\frac{\partial#1}{\partial#2}}
\newcommand{\innerprod}[2]{\left\langle#1 \cdot #2 \right\rangle}
\newcommand{\projection}[2]{\text{proj}_{#1}{#2}}

% summation 
\newcommand{\sumonen}{\sum_{i=1}^{n}}
\newcommand{\sumzeron}{\sum_{i=0}^{n}}
\newcommand{\sumoneinf}{\sum_{i=1}^{\infty}}
\newcommand{\sumzeroinf}{\sum_{i=0}^{\infty}}


\newcommand{\zerovec}{\mathbf{0}}
\newcommand{\vvec}{\mathbf{v}}
\newcommand{\uvec}{\mathbf{u}}
\newcommand{\wvec}{\mathbf{w}}

\newcommand{\vnorm}{\norm{\vvec}}
\newcommand{\unorm}{\norm{\uvec}}

\newcommand{\unitv}{\hat{\vvec}}
\newcommand{\unitu}{\hat{\uvec}}

\newcommand{\ibase}{\mathbf{\hat{i}}}
\newcommand{\jbase}{\mathbf{\hat{j}}}
\newcommand{\kbase}{\mathbf{\hat{k}}}


% Basic quantum states
\newcommand{\ketphi}{\ket{\phi}}
\newcommand{\braphi}{\bra{\phi}}
\newcommand{\ketpsi}{\ket{\psi}}
\newcommand{\brapsi}{\bra{\psi}}
\newcommand{\ketchi}{\ket{\chi}}
\newcommand{\brachi}{\bra{\chi}}

% Common quantum states
\newcommand{\ketzero}{\ket{0}}
\newcommand{\ketone}{\ket{1}}
\newcommand{\ketplus}{\ket{+}}
\newcommand{\ketminus}{\ket{-}}

% Bell states
\newcommand{\bellphi}{\ket{\Phi^+}}
\newcommand{\bellphim}{\ket{\Phi^-}}
\newcommand{\bellpsi}{\ket{\Psi^+}}
\newcommand{\bellpsim}{\ket{\Psi^-}}

% % quantum-commands.tex
% --- Created: December 2, 2025 ---

% =================================================================
% I. BASIC MATH FIELDS AND SPACES
% =================================================================

% A. Basic Fields (R, C, N, Z, Q)
\newcommand{\R}{\mathbb{R}} % Real numbers
\newcommand{\C}{\mathbb{C}} % Complex numbers
\newcommand{\N}{\mathbb{N}} % Natural numbers
\newcommand{\Z}{\mathbb{Z}} % Integers
\newcommand{\Q}{\mathbb{Q}} % Rational numbers

% B. Vector Spaces (R^n, C^n, etc.)
\newcommand{\Rn}{\mathbb{R}^n} % n-dim Real space
\newcommand{\Cn}{\mathbb{C}^n} % n-dim Complex space
\newcommand{\Nn}{\mathbb{N}^n}
\newcommand{\Zn}{\mathbb{Z}^n}
\newcommand{\Qn}{\mathbb{Q}^n}

\newcommand{\Rm}{\mathbb{R}^m} % n-dim Real space
\newcommand{\Cm}{\mathbb{C}^m} % n-dim Complex space

% C. Matrix Spaces (R^{m x n}, C^{n x n}, etc.)
\newcommand{\Rmn}{\mathbb{R}^{m \times n}} % Real m x n matrices
\newcommand{\Cmn}{\mathbb{C}^{m \times n}} % Complex m x n matrices
\newcommand{\Rnn}{\mathbb{R}^{n \times n}} % Real n x n matrices
\newcommand{\Cnn}{\mathbb{C}^{n \times n}} % Complex n x n matrices

\newcommand{\spanop}{\operatorname{span}}
\newcommand{\dimop}{\operatorname{dim}}
\newcommand{\trop}{\operatorname{tr}}

\newcommand{\mn}{{m \times n}}
\newcommand{\nn}{{n \times n}}
\newcommand{\nm}{{n \times m}}


% =================================================================
% II. VECTORS AND SCALAR OPERATIONS
% =================================================================

% A. Vector Symbols (Bolded)
\newcommand{\zerovec}{\mathbf{0}} % Zero vector
\newcommand{\vvec}{\mathbf{v}}    % Vector v
\newcommand{\uvec}{\mathbf{u}}    % Vector u
\newcommand{\wvec}{\mathbf{w}}    % Vector w
\newcommand{\rvec}{\mathbf{r}}    % Vector r
\newcommand{\cvec}{\mathbf{c}}    % Vector c (sometimes used for coefficient vector)

% B. Norms and Unit Vectors (Requires amsmath and maybe physics package for norm command)
\newcommand{\vnorm}{\norm{\vvec}}  % Norm of v
\newcommand{\unorm}{\norm{\uvec}}  % Norm of u
\newcommand{\unitv}{\hat{\vvec}}  % Unit vector v
\newcommand{\unitu}{\hat{\uvec}}  % Unit vector u

% C. Standard Basis Vectors (i, j, k)
\newcommand{\ibase}{\mathbf{\hat{i}}} % i-hat basis vector
\newcommand{\jbase}{\mathbf{\hat{j}}} % j-hat basis vector
\newcommand{\kbase}{\mathbf{\hat{k}}} % k-hat basis vector

% =================================================================
% III. COMMON MATH OPERATIONS
% =================================================================

% A. Calculus/Derivatives
\newcommand{\od}[2]{\frac{d#1}{d#2}} % Ordinary derivative
\newcommand{\pd}[2]{\frac{\partial#1}{\partial#2}} % Partial derivative

% B. Summation
\newcommand{\sumonen}{\sum_{i=1}^{n}} % Sum from 1 to n
\newcommand{\sumzeron}{\sum_{i=0}^{n}} % Sum from 0 to n
\newcommand{\sumoneinf}{\sum_{i=1}^{\infty}} % Sum from 1 to infinity
\newcommand{\sumzeroinf}{\sum_{i=0}^{\infty}} % Sum from 0 to infinity

% C. Vector Algebra
\newcommand{\innerprod}[2]{\left\langle#1 \cdot #2 \right\rangle} % Standard inner product (Dot product notation)
\newcommand{\projection}[2]{\text{proj}_{#1}{#2}} % Vector projection

% D. Conjugation/Adjoint
\newcommand{\conj}[1]{{#1}^*} % Complex conjugate (Often used for scalar or matrix elements)
\newcommand{\inverse}[1]{{#1}^{-1}}

% =================================================================
% IV. QUANTUM MECHANICS / BRA-KET NOTATION
% =================================================================

% A. Arbitrary States (Phi, Psi, Chi)
\newcommand{\ketphi}{\ket{\phi}} % Ket phi
\newcommand{\braphi}{\bra{\phi}} % Bra phi
\newcommand{\ketpsi}{\ket{\psi}} % Ket psi
\newcommand{\brapsi}{\bra{\psi}} % Bra psi
\newcommand{\ketchi}{\ket{\chi}} % Ket chi
\newcommand{\brachi}{\bra{\chi}} % Bra chi

\newcommand{\ketu}{\ket{u}}
\newcommand{\ketv}{\ket{v}}
\newcommand{\brau}{\bra{u}}
\newcommand{\brav}{\bra{v}}

% B. Common Single-Qubit States
\newcommand{\ketzero}{\ket{0}}   % Ket 0 (Computational basis)
\newcommand{\ketone}{\ket{1}}    % Ket 1 (Computational basis)
\newcommand{\ketplus}{\ket{+}}   % Ket + (Hadamard basis)
\newcommand{\ketminus}{\ket{-}}  % Ket - (Hadamard basis)

% C. Bell States (Maximally Entangled Two-Qubit States)
\newcommand{\bellphi}{\ket{\Phi^+}}  % Bell state Phi+
\newcommand{\bellphim}{\ket{\Phi^-}} % Bell state Phi-
\newcommand{\bellpsi}{\ket{\Psi^+}}  % Bell state Psi+
\newcommand{\bellpsim}{\ket{\Psi^-}} % Bell state Psi-

% D. Other notation:
\newcommand{\expectation}[1]{\left\langle {#1} \right\rangle }
\newcommand{\krondelta}[1]{\delta_{#1}}

\theoremstyle{plain}
\newtheorem{theorem}{Theorem}[section]
\newtheorem{definition}[theorem]{Definition}
\newtheorem{postulate}[theorem]{Postulate}
\newtheorem{corollary}[theorem]{Corollary}
\newtheorem{lemma}[theorem]{Lemma}

\newtheorem*{example}{Example}

\theoremstyle{definition}
\newtheorem*{solution}{Solution}

\title{\COURSECODE\ - \FULLCOURSENAME}
\author{\PROFESSORNAME\ - \STUDENTNAME}
\date{\SEMESTER\ Semester \YEAR}

\begin{document}

\maketitle

\hfill

The underpinnings of all scientific advancements is the ability to express natural phenomena with the art of Mathmatics; this is no different for the subject of Quantum Computing. While the boundaries of quantum computing have been pushed beyond limits in theoretical terms on university blackboards, it has become of great interest to realize the theoretical computational power with the advances of hardware and technology. 

However, these notes mainly concerns itself with the mathmatical underpinnings of quantum computing that the course surrounds itself with. \FULLCOURSENAME takes a scaffolding approach designed to efficiently convey the required theoretical understanding of mathmatics in order to able to learn quantum computing. As of writing, we are basing the notes on verison one of the textbook published in March 2025. In this text, we will primarily be using dirac notation for the expression of vectors, operators, and their interactions. 

\tableofcontents

\break


\section{Summation and Product Notations}

This section primarily focuses on the common notations applied across mathmatics to denote and shorten addition and product notation. 

\subsection{Summation over a single Variable}

The sigma notation is defined as follows

$$\sum_{i=1}^{n}f(i)$$

where we use sigma $\sum$ to represent the sum of a series. For example, the sum of all numbers in a series beginning with $m$ and ending at index $n$ is written as:

$$\sum_{i=m}^{n} a_i = a_m + a_{m+1} + a_{m+2} + \ldots + a_{n-1} + a_n$$

Sums can also be infinite, commonly seen when Sigma looks as follows: $\sum^{\infty}_{i=m}$. Infinite sums are either convergent or divergent. A few of the most common converging infinite sums are as follows:

$$\sum_{i=0}^{\infty} \frac{1}{2^i} = 1 + \frac{1}{2} + \frac{1}{4} + \ldots = 2$$
$$\sum_{i=0}^{\infty}  \frac{1}{i^2} = \frac{1}{1^2} + \frac{1}{2^2} + \frac{1}{3^2} + \ldots = \frac{\pi^2}{6}$$

The first example is an infinite geometric series, and the sum of the first $n$ terms is given by:

$$S_n = \sum_{i=0}^{n} \frac{1}{2^i} = \frac{1 - \frac{1}{2^n}}{1 - \frac{1}{2}}$$

As $n \rightarrow \infty, \frac{1}{2^n} \rightarrow 0$. Consequently, $S_n \rightarrow \frac{1}{1 - \frac{1}{2}} = 2$. A rigorous proof of the second example requires extensive calculus and is not immediately obvious. While any mathmatical symbol can be used for the index of a summation, it is more practical to use something other than $i$ as in the context of complex numbers, $i$ commonly denotes the complex number$\sqrt{-1}$. moreover, sume can also be specified using descriptions. For example, 

$$\sum_{p \in P} f(p) \qquad P \in \mathbb{N^\prime}$$

where $\mathbb{N^\prime}$ is the set of all prime numbers. Summations can also contain parameters other than the index, which results in functions of those parameters. For example the discrete Fourier transform (DFT) is given by

$$\tilde{x}_k = \frac{1}{\sqrt{N}} \sum_{n=0}^{N - 1} x_n e^{- \frac{2 \pi i}{N} kn}, \quad k = 0, 1, \ldots N-1$$

where $x_n$ represents the $N$ values index by $n$ and $\tilde{x}_k$ are the Fourier coefficients. Here, $i$ is the imaginary numebr and $N$ is a positive integer representing the dimension fo the DFT, of which we will cover in greater depth in Chapter 3. The following are some useful summation forumae commonly encountered in quantum computing:

\[
\sum_{i=1}^{n} i = \frac{n(n+1)}{2}
\]

\[
\sum_{i=1}^{n} i^2 = \frac{n(n+1)(2n+1)}{6}
\]

\[
\sum_{i=1}^{n} i^3 = \left(\frac{n(n+1)}{2}\right)^2
\]

\[
\sum_{i=0}^{n} \left(a_0 + id\right) = (n+1)\left(a_0 + \frac{nd}{2}\right) \quad \text{(arithmetic series)}
\]

\[
\sum_{i=0}^{n} a^i = \frac{1 - a^{n+1}}{1 - a} \quad \text{(geometric series)}
\]

\[
(a + b)^n = \sum_{i=0}^{n} \binom{n}{i} a^{n-i} b^i \quad \text{(binomial theorem)}
\]

\[
\frac{1}{1 - x} = \sum_{n=0}^{\infty} x^n = 1 + x + x^2 + x^3 + \cdots \quad (\lvert x \rvert < 1)
\]

\[
\frac{1}{(1-x)^2} = \sum_{n=1}^{\infty} n x^{n-1} = 1 + 2x + 3x^2 + 4x^3 + \cdots \quad (\lvert x \rvert < 1)
\]

\[
\ln(1 + x) = \sum_{n=1}^{\infty} \frac{(-1)^{n+1}}{n} x^n = x - \frac{x^2}{2} + \frac{x^3}{3} - \cdots \quad (\lvert x \rvert < 1)
\]

\[
e^x = \sum_{n=0}^{\infty} \frac{x^n}{n!} = 1 + x + \frac{x^2}{2!} + \frac{x^3}{3!} + \cdots
\]

\[
\sin x = \sum_{n=0}^{\infty} \frac{(-1)^n}{(2n+1)!} x^{2n+1} = x - \frac{x^3}{3!} + \frac{x^5}{5!} - \cdots
\]

\[
\cos x = \sum_{n=0}^{\infty} \frac{(-1)^n}{(2n)!} x^{2n} = 1 - \frac{x^2}{2!} + \frac{x^4}{4!} - \cdots
\]

 


\end{document}